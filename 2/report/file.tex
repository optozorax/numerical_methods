\mytitlepage{прикладной математики}{2}{Численные методы}{Итерационные методы решения СЛАУ}{ПМ-63}{Шепрут И.И.}{11}{Задорожный А.Г.}{2018}

\section{Цель работы}

Разработать программы решения СЛАУ методами Якоби, Гаусса-Зейделя с хранением матрицы в диагональном формате. Исследовать сходимость методов для различных тестовых матриц и её зависимость от параметра релаксации. Изучить возможность оценки порядка числа обусловленности матрицы путем вычислительного эксперимента.

\textbf{Вариант 11:} 7-ми диагональная матрица c параметрами $m$, $k$ --- количество нулевых диагоналей, $n$ --- размерность матрицы.

\section{Код программы}

Программа состоит из нескольких частей:
\noindent\begin{easylist}
\ListProperties(Hang1=true, Margin2=12pt, Style1**=$\bullet$ , Hide2=1, Hide1=1)
& То, что было в прошлом отчете и здесь не приводится:
&& \texttt{common.h + common.cpp} --- пара общих функций и объявление вещественных типов.
&& \texttt{matrix.h + matrix.cpp} --- модуль для работы с матрицами в плотном формате.
&& \texttt{vector.h + vector.cpp} --- модуль для работы с векторами.
& Новый код:
&& \texttt{diagonal.h + diagonal.cpp} --- модуль для работы с матрицами в диагональном формате.
&& \texttt{table\_generator.cpp} --- программа, которая генерирует таблицы.
&& \texttt{diagonal\_test.cpp} --- юнит-тестирование модуля для работы с диагональными матрицами.
\end{easylist}

\begin{multicols*}{2}
\mycodeinput{c++}{nm/2/diagonal.h}{diagonal.h}
\mycodeinput{c++}{nm/2/diagonal.cpp}{diagonal.cpp}
\mycodeinput{c++}{nm/2/numerical_tests.cpp}{table\_generator.cpp}
\end{multicols*}
\mycodeinput{c++}{nm/2/diagonal_test.cpp}{diagonal\_test.cpp}

\section{Тестирование}

Для тестирования использовалось юнит-тестирование и библиотека Catch. Было протестировано получение необходимой относительной невязки на матрицах с диагональным преобладанием.

\begin{center}
\noindent\includegraphics[scale=0.7]{unit_test.png}
\end{center}

\section{Исследования}

\subsection{Матрица с диагональным преобладанием}
$$ A=\left(\quad\begin{matrix}
\cellcolor{green!30}2 & \cellcolor{green!30}0 & 0 & 0 & \cellcolor{green!30}0 & 0 & \cellcolor{green!30}-1 & 0 & 0 & 0 \\
\cellcolor{green!30}-3 & \cellcolor{green!30}13 & \cellcolor{green!30}-4 & 0 & 0 & \cellcolor{green!30}-4 & 0 & \cellcolor{green!30}-2 & 0 & 0 \\
0 & \cellcolor{green!30}0 & \cellcolor{green!30}7 & \cellcolor{green!30}-3 & 0 & 0 & \cellcolor{green!30}-2 & 0 & \cellcolor{green!30}-2 & 0 \\
0 & 0 & \cellcolor{green!30}-3 & \cellcolor{green!30}8 & \cellcolor{green!30}-2 & 0 & 0 & \cellcolor{green!30}0 & 0 & \cellcolor{green!30}-3 \\
\cellcolor{green!30}-2 & 0 & 0 & \cellcolor{green!30}-2 & \cellcolor{green!30}5 & \cellcolor{green!30}-1 & 0 & 0 & \cellcolor{green!30}0 & 0 \\
0 & \cellcolor{green!30}-1 & 0 & 0 & \cellcolor{green!30}-1 & \cellcolor{green!30}2 & \cellcolor{green!30}0 & 0 & 0 & \cellcolor{green!30}0 \\
\cellcolor{green!30}-2 & 0 & \cellcolor{green!30}-4 & 0 & 0 & \cellcolor{green!30}0 & \cellcolor{green!30}6 & \cellcolor{green!30}0 & 0 & 0 \\
0 & \cellcolor{green!30}0 & 0 & \cellcolor{green!30}-3 & 0 & 0 & \cellcolor{green!30}-3 & \cellcolor{green!30}7 & \cellcolor{green!30}-1 & 0 \\
0 & 0 & \cellcolor{green!30}-2 & 0 & \cellcolor{green!30}-4 & 0 & 0 & \cellcolor{green!30}-3 & \cellcolor{green!30}9 & \cellcolor{green!30}0 \\
0 & 0 & 0 & \cellcolor{green!30}-1 & 0 & \cellcolor{green!30}-4 & 0 & 0 & \cellcolor{green!30}-4 & \cellcolor{green!30}9 
\end{matrix}\quad\right), X=\begin{pmatrix}1 \\
2 \\
3 \\
4 \\
5 \\
6 \\
7 \\
8 \\
9 \\
10 
\end{pmatrix}, F=\begin{pmatrix}-5 \\
-29 \\
-23 \\
-17 \\
9 \\
5 \\
28 \\
14 \\
31 \\
26 
\end{pmatrix} $$

$$ \varepsilon = 10^{-14}, \quad iterations_{max} = 100000, \quad start = \begin{pmatrix} 0 & 0 & 0 & 0 & 0 & 0 & 0 & 0 & 0 & 0 \end{pmatrix}^T $$

\setlength{\tabcolsep}{2pt}
\tabulinesep=0.3mm
\noindent{\scriptsize\texttt{\begin{longtabu}{
|X[-1,c]||X[-1,c]|X[-1,c]|X[-1,c]|X[-1,c]|X[-1,c]|
p{0.05cm}
|X[-1,c]||X[-1,c]|X[-1,c]|X[-1,c]|X[-1,c]|X[-1,c]|}
\cline{1-6}\cline{8-13}
\multicolumn{6}{|c|}{Метод Якоби} && \multicolumn{6}{c|}{Метод Зейделя} \\
\cline{1-6}\cline{8-13}
0.00 & \tiny{\tcell{0.0000000000000000 \\ 0.0000000000000000 \\ 0.0000000000000000 \\ 0.0000000000000000 \\ 0.0000000000000000 \\ 0.0000000000000000 \\ 0.0000000000000000 \\ 0.0000000000000000 \\ 0.0000000000000000 \\ 0.0000000000000000}} & \tiny{\tcell{1.00e+00 \\ 2.00e+00 \\ 3.00e+00 \\ 4.00e+00 \\ 5.00e+00 \\ 6.00e+00 \\ 7.00e+00 \\ 8.00e+00 \\ 9.00e+00 \\ 1.00e+01}} & 1.00e+00 & 1.00 & 100000 &  & 0.00 & \tiny{\tcell{0.0000000000000000 \\ 0.0000000000000000 \\ 0.0000000000000000 \\ 0.0000000000000000 \\ 0.0000000000000000 \\ 0.0000000000000000 \\ 0.0000000000000000 \\ 0.0000000000000000 \\ 0.0000000000000000 \\ 0.0000000000000000}} & \tiny{\tcell{1.0e+00 \\ 2.0e+00 \\ 3.0e+00 \\ 4.0e+00 \\ 5.0e+00 \\ 6.0e+00 \\ 7.0e+00 \\ 8.0e+00 \\ 9.0e+00 \\ 1.0e+01}} & 1.00e+00 & 1.00 & 100000 \\
\cline{1-6}\cline{8-13}
0.10 & \tiny{\tcell{0.9999999999997312 \\ 1.9999999999994800 \\ 2.9999999999994125 \\ 3.9999999999994142 \\ 4.9999999999995302 \\ 5.9999999999994751 \\ 6.9999999999994902 \\ 7.9999999999994200 \\ 8.9999999999994404 \\ 9.9999999999994262}} & \tiny{\tcell{2.69e-13 \\ 5.20e-13 \\ 5.88e-13 \\ 5.86e-13 \\ 4.70e-13 \\ 5.25e-13 \\ 5.10e-13 \\ 5.80e-13 \\ 5.60e-13 \\ 5.74e-13}} & 9.93e-15 & 8.54 & 5678 &  & 0.10 & \tiny{\tcell{0.9999999999997365 \\ 1.9999999999994895 \\ 2.9999999999994222 \\ 3.9999999999994245 \\ 4.9999999999995381 \\ 5.9999999999994866 \\ 6.9999999999995008 \\ 7.9999999999994333 \\ 8.9999999999994529 \\ 9.9999999999994351}} & \tiny{\tcell{2.6e-13 \\ 5.1e-13 \\ 5.8e-13 \\ 5.8e-13 \\ 4.6e-13 \\ 5.1e-13 \\ 5.0e-13 \\ 5.7e-13 \\ 5.5e-13 \\ 5.6e-13}} & 9.95e-15 & 8.36 & 5381 \\
\cline{1-6}\cline{8-13}
0.20 & \tiny{\tcell{0.9999999999997413 \\ 1.9999999999994982 \\ 2.9999999999994329 \\ 3.9999999999994347 \\ 4.9999999999995453 \\ 5.9999999999994946 \\ 6.9999999999995097 \\ 7.9999999999994413 \\ 8.9999999999994564 \\ 9.9999999999994404}} & \tiny{\tcell{2.59e-13 \\ 5.02e-13 \\ 5.67e-13 \\ 5.65e-13 \\ 4.55e-13 \\ 5.05e-13 \\ 4.90e-13 \\ 5.59e-13 \\ 5.44e-13 \\ 5.60e-13}} & 9.96e-15 & 8.22 & 2833 &  & 0.20 & \tiny{\tcell{0.9999999999997410 \\ 1.9999999999994977 \\ 2.9999999999994329 \\ 3.9999999999994373 \\ 4.9999999999995479 \\ 5.9999999999994991 \\ 6.9999999999995124 \\ 7.9999999999994458 \\ 8.9999999999994635 \\ 9.9999999999994511}} & \tiny{\tcell{2.6e-13 \\ 5.0e-13 \\ 5.7e-13 \\ 5.6e-13 \\ 4.5e-13 \\ 5.0e-13 \\ 4.9e-13 \\ 5.5e-13 \\ 5.4e-13 \\ 5.5e-13}} & 9.93e-15 & 8.19 & 2532 \\
\cline{1-6}\cline{8-13}
0.30 & \tiny{\tcell{0.9999999999997444 \\ 1.9999999999995042 \\ 2.9999999999994396 \\ 3.9999999999994413 \\ 4.9999999999995506 \\ 5.9999999999995008 \\ 6.9999999999995159 \\ 7.9999999999994476 \\ 8.9999999999994635 \\ 9.9999999999994476}} & \tiny{\tcell{2.56e-13 \\ 4.96e-13 \\ 5.60e-13 \\ 5.59e-13 \\ 4.49e-13 \\ 4.99e-13 \\ 4.84e-13 \\ 5.52e-13 \\ 5.36e-13 \\ 5.52e-13}} & 9.86e-15 & 8.21 & 1884 &  & 0.30 & \tiny{\tcell{0.9999999999997394 \\ 1.9999999999994948 \\ 2.9999999999994307 \\ 3.9999999999994369 \\ 4.9999999999995488 \\ 5.9999999999994991 \\ 6.9999999999995115 \\ 7.9999999999994467 \\ 8.9999999999994653 \\ 9.9999999999994529}} & \tiny{\tcell{2.6e-13 \\ 5.1e-13 \\ 5.7e-13 \\ 5.6e-13 \\ 4.5e-13 \\ 5.0e-13 \\ 4.9e-13 \\ 5.5e-13 \\ 5.3e-13 \\ 5.5e-13}} & 1.00e-14 & 8.14 & 1582 \\
\cline{1-6}\cline{8-13}
0.40 & \tiny{\tcell{0.9999999999997434 \\ 1.9999999999995017 \\ 2.9999999999994365 \\ 3.9999999999994382 \\ 4.9999999999995488 \\ 5.9999999999994991 \\ 6.9999999999995133 \\ 7.9999999999994449 \\ 8.9999999999994618 \\ 9.9999999999994458}} & \tiny{\tcell{2.57e-13 \\ 4.98e-13 \\ 5.64e-13 \\ 5.62e-13 \\ 4.51e-13 \\ 5.01e-13 \\ 4.87e-13 \\ 5.55e-13 \\ 5.38e-13 \\ 5.54e-13}} & 9.90e-15 & 8.21 & 1409 &  & 0.40 & \tiny{\tcell{0.9999999999997397 \\ 1.9999999999994955 \\ 2.9999999999994316 \\ 3.9999999999994396 \\ 4.9999999999995506 \\ 5.9999999999995026 \\ 6.9999999999995142 \\ 7.9999999999994511 \\ 8.9999999999994689 \\ 9.9999999999994582}} & \tiny{\tcell{2.6e-13 \\ 5.0e-13 \\ 5.7e-13 \\ 5.6e-13 \\ 4.5e-13 \\ 5.0e-13 \\ 4.9e-13 \\ 5.5e-13 \\ 5.3e-13 \\ 5.4e-13}} & 9.99e-15 & 8.10 & 1107 \\
\cline{1-6}\cline{8-13}
0.50 & \tiny{\tcell{0.9999999999997422 \\ 1.9999999999994995 \\ 2.9999999999994347 \\ 3.9999999999994369 \\ 4.9999999999995470 \\ 5.9999999999994973 \\ 6.9999999999995115 \\ 7.9999999999994422 \\ 8.9999999999994582 \\ 9.9999999999994440}} & \tiny{\tcell{2.58e-13 \\ 5.00e-13 \\ 5.65e-13 \\ 5.63e-13 \\ 4.53e-13 \\ 5.03e-13 \\ 4.88e-13 \\ 5.58e-13 \\ 5.42e-13 \\ 5.56e-13}} & 9.94e-15 & 8.21 & 1124 &  & 0.50 & \tiny{\tcell{0.9999999999997498 \\ 1.9999999999995155 \\ 2.9999999999994547 \\ 3.9999999999994644 \\ 4.9999999999995710 \\ 5.9999999999995257 \\ 6.9999999999995355 \\ 7.9999999999994778 \\ 8.9999999999994955 \\ 9.9999999999994866}} & \tiny{\tcell{2.5e-13 \\ 4.8e-13 \\ 5.5e-13 \\ 5.4e-13 \\ 4.3e-13 \\ 4.7e-13 \\ 4.6e-13 \\ 5.2e-13 \\ 5.0e-13 \\ 5.1e-13}} & 9.76e-15 & 7.92 & 823 \\
\cline{1-6}\cline{8-13}
0.60 & \tiny{\tcell{0.9999999999997411 \\ 1.9999999999994975 \\ 2.9999999999994320 \\ 3.9999999999994347 \\ 4.9999999999995453 \\ 5.9999999999994946 \\ 6.9999999999995097 \\ 7.9999999999994404 \\ 8.9999999999994564 \\ 9.9999999999994422}} & \tiny{\tcell{2.59e-13 \\ 5.02e-13 \\ 5.68e-13 \\ 5.65e-13 \\ 4.55e-13 \\ 5.05e-13 \\ 4.90e-13 \\ 5.60e-13 \\ 5.44e-13 \\ 5.58e-13}} & 9.89e-15 & 8.29 & 934 &  & 0.60 & \tiny{\tcell{0.9999999999997568 \\ 1.9999999999995293 \\ 2.9999999999994706 \\ 3.9999999999994831 \\ 4.9999999999995870 \\ 5.9999999999995435 \\ 6.9999999999995524 \\ 7.9999999999994973 \\ 8.9999999999995168 \\ 9.9999999999995097}} & \tiny{\tcell{2.4e-13 \\ 4.7e-13 \\ 5.3e-13 \\ 5.2e-13 \\ 4.1e-13 \\ 4.6e-13 \\ 4.5e-13 \\ 5.0e-13 \\ 4.8e-13 \\ 4.9e-13}} & 9.68e-15 & 7.69 & 633 \\
\cline{1-6}\cline{8-13}
0.70 & \tiny{\tcell{0.9999999999997469 \\ 1.9999999999995088 \\ 2.9999999999994449 \\ 3.9999999999994476 \\ 4.9999999999995559 \\ 5.9999999999995062 \\ 6.9999999999995204 \\ 7.9999999999994520 \\ 8.9999999999994689 \\ 9.9999999999994547}} & \tiny{\tcell{2.53e-13 \\ 4.91e-13 \\ 5.55e-13 \\ 5.52e-13 \\ 4.44e-13 \\ 4.94e-13 \\ 4.80e-13 \\ 5.48e-13 \\ 5.31e-13 \\ 5.45e-13}} & 9.64e-15 & 8.31 & 799 &  & 0.70 & \tiny{\tcell{0.9999999999997663 \\ 1.9999999999995477 \\ 2.9999999999994920 \\ 3.9999999999995066 \\ 4.9999999999996056 \\ 5.9999999999995648 \\ 6.9999999999995719 \\ 7.9999999999995230 \\ 8.9999999999995399 \\ 9.9999999999995346}} & \tiny{\tcell{2.3e-13 \\ 4.5e-13 \\ 5.1e-13 \\ 4.9e-13 \\ 3.9e-13 \\ 4.4e-13 \\ 4.3e-13 \\ 4.8e-13 \\ 4.6e-13 \\ 4.7e-13}} & 9.53e-15 & 7.46 & 497 \\
\cline{1-6}\cline{8-13}
0.80 & \tiny{\tcell{0.9999999999997445 \\ 1.9999999999995040 \\ 2.9999999999994396 \\ 3.9999999999994413 \\ 4.9999999999995515 \\ 5.9999999999995008 \\ 6.9999999999995159 \\ 7.9999999999994476 \\ 8.9999999999994618 \\ 9.9999999999994493}} & \tiny{\tcell{2.55e-13 \\ 4.96e-13 \\ 5.60e-13 \\ 5.59e-13 \\ 4.49e-13 \\ 4.99e-13 \\ 4.84e-13 \\ 5.52e-13 \\ 5.38e-13 \\ 5.51e-13}} & 9.83e-15 & 8.23 & 697 &  & 0.80 & \tiny{\tcell{0.9999999999997817 \\ 1.9999999999995768 \\ 2.9999999999995257 \\ 3.9999999999995417 \\ 4.9999999999996350 \\ 5.9999999999995977 \\ 6.9999999999996039 \\ 7.9999999999995604 \\ 8.9999999999995772 \\ 9.9999999999995737}} & \tiny{\tcell{2.2e-13 \\ 4.2e-13 \\ 4.7e-13 \\ 4.6e-13 \\ 3.7e-13 \\ 4.0e-13 \\ 4.0e-13 \\ 4.4e-13 \\ 4.2e-13 \\ 4.3e-13}} & 9.42e-15 & 6.99 & 395 \\
\cline{1-6}\cline{8-13}
0.90 & \tiny{\tcell{0.9999999999997456 \\ 1.9999999999995062 \\ 2.9999999999994418 \\ 3.9999999999994440 \\ 4.9999999999995532 \\ 5.9999999999995035 \\ 6.9999999999995186 \\ 7.9999999999994502 \\ 8.9999999999994671 \\ 9.9999999999994511}} & \tiny{\tcell{2.54e-13 \\ 4.94e-13 \\ 5.58e-13 \\ 5.56e-13 \\ 4.47e-13 \\ 4.96e-13 \\ 4.81e-13 \\ 5.50e-13 \\ 5.33e-13 \\ 5.49e-13}} & 9.85e-15 & 8.17 & 618 &  & 0.90 & \tiny{\tcell{0.9999999999997934 \\ 1.9999999999996005 \\ 2.9999999999995537 \\ 3.9999999999995723 \\ 4.9999999999996607 \\ 5.9999999999996261 \\ 6.9999999999996296 \\ 7.9999999999995932 \\ 8.9999999999996092 \\ 9.9999999999996074}} & \tiny{\tcell{2.1e-13 \\ 4.0e-13 \\ 4.5e-13 \\ 4.3e-13 \\ 3.4e-13 \\ 3.7e-13 \\ 3.7e-13 \\ 4.1e-13 \\ 3.9e-13 \\ 3.9e-13}} & 9.54e-15 & 6.43 & 315 \\
\cline{1-6}\cline{8-13}
1.00 & \tiny{\tcell{0.9999999999997509 \\ 1.9999999999995151 \\ 2.9999999999994529 \\ 3.9999999999994547 \\ 4.9999999999995612 \\ 5.9999999999995133 \\ 6.9999999999995266 \\ 7.9999999999994609 \\ 8.9999999999994760 \\ 9.9999999999994618}} & \tiny{\tcell{2.49e-13 \\ 4.85e-13 \\ 5.47e-13 \\ 5.45e-13 \\ 4.39e-13 \\ 4.87e-13 \\ 4.73e-13 \\ 5.39e-13 \\ 5.24e-13 \\ 5.38e-13}} & 9.70e-15 & 8.14 & 555 &  & 1.00 & \tiny{\tcell{0.9999999999998201 \\ 1.9999999999996521 \\ 2.9999999999996123 \\ 3.9999999999996323 \\ 4.9999999999997087 \\ 5.9999999999996803 \\ 6.9999999999996820 \\ 7.9999999999996536 \\ 8.9999999999996678 \\ 9.9999999999996696}} & \tiny{\tcell{1.8e-13 \\ 3.5e-13 \\ 3.9e-13 \\ 3.7e-13 \\ 2.9e-13 \\ 3.2e-13 \\ 3.2e-13 \\ 3.5e-13 \\ 3.3e-13 \\ 3.3e-13}} & 9.02e-15 & 5.84 & 251 \\
\cline{1-6}\cline{8-13}
\cellcolor{green!30}{1.08} & \tiny{\cellcolor{green!30}{\tcell{0.9999999999997538 \\ 1.9999999999995233 \\ 2.9999999999994595 \\ 3.9999999999994640 \\ 4.9999999999995683 \\ 5.9999999999995204 \\ 6.9999999999995355 \\ 7.9999999999994680 \\ 8.9999999999994831 \\ 9.9999999999994706}}} & \tiny{\cellcolor{green!30}{\tcell{2.5e-13 \\ 4.8e-13 \\ 5.4e-13 \\ 5.4e-13 \\ 4.3e-13 \\ 4.8e-13 \\ 4.6e-13 \\ 5.3e-13 \\ 5.2e-13 \\ 5.3e-13}}} & \cellcolor{green!30}{9.3e-15} & \cellcolor{green!30}{8.35} & \cellcolor{green!30}{513} &  & 1.08 & \tiny{\tcell{0.9999999999998229 \\ 1.9999999999996569 \\ 2.9999999999996190 \\ 3.9999999999996430 \\ 4.9999999999997176 \\ 5.9999999999996909 \\ 6.9999999999996900 \\ 7.9999999999996643 \\ 8.9999999999996803 \\ 9.9999999999996856}} & \tiny{\tcell{1.8e-13 \\ 3.4e-13 \\ 3.8e-13 \\ 3.6e-13 \\ 2.8e-13 \\ 3.1e-13 \\ 3.1e-13 \\ 3.4e-13 \\ 3.2e-13 \\ 3.1e-13}} & 9.64e-15 & 5.30 & 207 \\
\cline{1-6}\cline{8-13}
1.10 & \tiny{\tcell{1.0000000000000113 \\ 1.9999999999999800 \\ 3.0000000000000013 \\ 4.0000000000000124 \\ 4.9999999999999813 \\ 6.0000000000000284 \\ 6.9999999999999893 \\ 8.0000000000000053 \\ 9.0000000000000053 \\ 9.9999999999999822}} & \tiny{\tcell{-1.13e-14 \\ 2.00e-14 \\ -1.33e-15 \\ -1.24e-14 \\ 1.87e-14 \\ -2.84e-14 \\ 1.07e-14 \\ -5.33e-15 \\ -5.33e-15 \\ 1.78e-14}} & 9.12e-15 & 0.27 & 875 &  & 1.10 & \tiny{\tcell{0.9999999999998258 \\ 1.9999999999996636 \\ 2.9999999999996265 \\ 3.9999999999996509 \\ 4.9999999999997247 \\ 5.9999999999996989 \\ 6.9999999999996971 \\ 7.9999999999996723 \\ 8.9999999999996891 \\ 9.9999999999996927}} & \tiny{\tcell{1.7e-13 \\ 3.4e-13 \\ 3.7e-13 \\ 3.5e-13 \\ 2.8e-13 \\ 3.0e-13 \\ 3.0e-13 \\ 3.3e-13 \\ 3.1e-13 \\ 3.1e-13}} & 9.63e-15 & 5.19 & 197 \\
\cline{1-6}\cline{8-13}
& & & & & & & 1.20 & \tiny{\tcell{0.9999999999998664 \\ 1.9999999999997420 \\ 2.9999999999997149 \\ 3.9999999999997380 \\ 4.9999999999997948 \\ 5.9999999999997771 \\ 6.9999999999997735 \\ 7.9999999999997611 \\ 8.9999999999997744 \\ 9.9999999999997797}} & \tiny{\tcell{1.3e-13 \\ 2.6e-13 \\ 2.9e-13 \\ 2.6e-13 \\ 2.1e-13 \\ 2.2e-13 \\ 2.3e-13 \\ 2.4e-13 \\ 2.3e-13 \\ 2.2e-13}} & 8.69e-15 & 4.28 & 152 \\
\cline{1-6}\cline{8-13}
& & & & & & & 1.30 & \tiny{\tcell{0.9999999999998852 \\ 1.9999999999997773 \\ 2.9999999999997558 \\ 3.9999999999997824 \\ 4.9999999999998312 \\ 5.9999999999998179 \\ 6.9999999999998126 \\ 7.9999999999998055 \\ 8.9999999999998206 \\ 9.9999999999998295}} & \tiny{\tcell{1.1e-13 \\ 2.2e-13 \\ 2.4e-13 \\ 2.2e-13 \\ 1.7e-13 \\ 1.8e-13 \\ 1.9e-13 \\ 1.9e-13 \\ 1.8e-13 \\ 1.7e-13}} & 9.30e-15 & 3.31 & 111 \\
\cline{1-6}\cline{8-13}
& & & & & & & \cellcolor{green!30}{1.37} & \tiny{\cellcolor{green!30}{\tcell{0.9999999999999157 \\ 1.9999999999998914 \\ 2.9999999999998836 \\ 3.9999999999999565 \\ 4.9999999999999458 \\ 5.9999999999998996 \\ 6.9999999999998996 \\ 7.9999999999999503 \\ 8.9999999999999432 \\ 9.9999999999999130}}} & \tiny{\cellcolor{green!30}{\tcell{8.4e-14 \\ 1.1e-13 \\ 1.2e-13 \\ 4.4e-14 \\ 5.4e-14 \\ 1.0e-13 \\ 1.0e-13 \\ 5.0e-14 \\ 5.7e-14 \\ 8.7e-14}}} & \cellcolor{green!30}{9.10e-15} & \cellcolor{green!30}{1.49} & \cellcolor{green!30}{85} \\
\cline{1-6}\cline{8-13}
& & & & & & & 1.40 & \tiny{\tcell{1.0000000000000024 \\ 1.9999999999999885 \\ 3.0000000000000351 \\ 3.9999999999999760 \\ 4.9999999999999698 \\ 5.9999999999999920 \\ 7.0000000000000480 \\ 8.0000000000000018 \\ 8.9999999999999876 \\ 10.0000000000000000}} & \tiny{\tcell{-2.4e-15 \\ 1.2e-14 \\ -3.5e-14 \\ 2.4e-14 \\ 3.0e-14 \\ 8.0e-15 \\ -4.8e-14 \\ -1.8e-15 \\ 1.2e-14 \\ 0.0e+00}} & 7.35e-15 & 0.51 & 103 \\
\cline{1-6}\cline{8-13}
& & & & & & & 1.50 & \tiny{\tcell{0.9999999999999558 \\ 1.9999999999999600 \\ 3.0000000000000373 \\ 4.0000000000000036 \\ 4.9999999999999520 \\ 5.9999999999999396 \\ 7.0000000000000524 \\ 8.0000000000000355 \\ 8.9999999999999840 \\ 9.9999999999999556}} & \tiny{\tcell{4.4e-14 \\ 4.0e-14 \\ -3.7e-14 \\ -3.6e-15 \\ 4.8e-14 \\ 6.0e-14 \\ -5.2e-14 \\ -3.6e-14 \\ 1.6e-14 \\ 4.4e-14}} & 8.49e-15 & 0.79 & 285 \\
\cline{1-6}\cline{8-13}
\end{longtabu}}}

\noindent\begin{tikzpicture}
\begin{semilogyaxis}[xlabel=w,ylabel=Iterations,width=\textwidth, height=6cm]
\addplot[red, no markers] table [y=it1, x=w1]{A.dat};
\addplot[blue, no markers] table [y=it2, x=w2]{A.dat};
\legend{Jacobi,Seidel}
\end{semilogyaxis}
\end{tikzpicture}

\subsubsection{Расчет числа обусловленности через MathCad}
$\displaystyle
	\begin{aligned}
		&\mathop{conde}(A) = 133.86 \\
		&\mathop{condi}(A) = 111.728 \\
		&\mathop{cond1}(A) = 200  \\
		&\mathop{cond2}(A) = 74.622  \\
		&\frac{\lambda^A_{max}}{\lambda^A_{min}} = \frac{13.709}{0.278} = 49.313  \\
	\end{aligned}
$

\subsection{Матрица с обратным знаком внедиагональных элементов}
$$ B=\left(\quad\begin{matrix}
\cellcolor{green!30}2 & \cellcolor{green!30}0 & 0 & 0 & \cellcolor{green!30}0 & 0 & \cellcolor{green!30}1 & 0 & 0 & 0 \\
\cellcolor{green!30}3 & \cellcolor{green!30}13 & \cellcolor{green!30}4 & 0 & 0 & \cellcolor{green!30}4 & 0 & \cellcolor{green!30}2 & 0 & 0 \\
0 & \cellcolor{green!30}0 & \cellcolor{green!30}7 & \cellcolor{green!30}3 & 0 & 0 & \cellcolor{green!30}2 & 0 & \cellcolor{green!30}2 & 0 \\
0 & 0 & \cellcolor{green!30}3 & \cellcolor{green!30}8 & \cellcolor{green!30}2 & 0 & 0 & \cellcolor{green!30}0 & 0 & \cellcolor{green!30}3 \\
\cellcolor{green!30}2 & 0 & 0 & \cellcolor{green!30}2 & \cellcolor{green!30}5 & \cellcolor{green!30}1 & 0 & 0 & \cellcolor{green!30}0 & 0 \\
0 & \cellcolor{green!30}1 & 0 & 0 & \cellcolor{green!30}1 & \cellcolor{green!30}2 & \cellcolor{green!30}0 & 0 & 0 & \cellcolor{green!30}0 \\
\cellcolor{green!30}2 & 0 & \cellcolor{green!30}4 & 0 & 0 & \cellcolor{green!30}0 & \cellcolor{green!30}6 & \cellcolor{green!30}0 & 0 & 0 \\
0 & \cellcolor{green!30}0 & 0 & \cellcolor{green!30}3 & 0 & 0 & \cellcolor{green!30}3 & \cellcolor{green!30}7 & \cellcolor{green!30}1 & 0 \\
0 & 0 & \cellcolor{green!30}2 & 0 & \cellcolor{green!30}4 & 0 & 0 & \cellcolor{green!30}3 & \cellcolor{green!30}9 & \cellcolor{green!30}0 \\
0 & 0 & 0 & \cellcolor{green!30}1 & 0 & \cellcolor{green!30}4 & 0 & 0 & \cellcolor{green!30}4 & \cellcolor{green!30}9 
\end{matrix}\quad\right), X=\begin{pmatrix}1 \\
2 \\
3 \\
4 \\
5 \\
6 \\
7 \\
8 \\
9 \\
10 
\end{pmatrix}, F=\begin{pmatrix}9 \\
81 \\
65 \\
81 \\
41 \\
19 \\
56 \\
98 \\
131 \\
154 
\end{pmatrix} $$

$$ \varepsilon = 10^{-14}, \quad iterations_{max} = 100000, \quad start = \begin{pmatrix} 0 & 0 & 0 & 0 & 0 & 0 & 0 & 0 & 0 & 0 \end{pmatrix}^T $$

\setlength{\tabcolsep}{2pt}
\tabulinesep=0.3mm
\noindent{\scriptsize\texttt{\begin{longtabu}{
|X[-1,c]||X[-1,c]|X[-1,c]|X[-1,c]|X[-1,c]|X[-1,c]|
p{0.05cm}
|X[-1,c]||X[-1,c]|X[-1,c]|X[-1,c]|X[-1,c]|X[-1,c]|}
\cline{1-6}\cline{8-13}
\multicolumn{6}{|c|}{Метод Якоби} && \multicolumn{6}{c|}{Метод Зейделя} \\
\hhline{*{6}{-}~*{6}{-}}
$w$ & $x$ & $x-x^*$ & \tcell{\tiny Относительная\\\tiny невязка} & {\tiny $\mathop{cond}(A) >$} & {\tiny Итераций} && $w$ & $x$ & $x-x^*$ & \tcell{\tiny Относительная\\\tiny невязка} & {\tiny $\mathop{cond}(A) >$} & {\tiny Итераций} \\
\hhline{*{6}{-}~*{6}{-}}
0.10 & \tiny{\tcell{0.9999999999999448 \\ 1.9999999999999298 \\ 2.9999999999995093 \\ 4.0000000000006404 \\ 4.9999999999995248 \\ 6.0000000000004841 \\ 7.0000000000002656 \\ 7.9999999999995683 \\ 9.0000000000005063 \\ 9.9999999999993410}} & \tiny{\tcell{5.5e-14 \\ 7.0e-14 \\ 4.9e-13 \\ -6.4e-13 \\ 4.8e-13 \\ -4.8e-13 \\ -2.7e-13 \\ 4.3e-13 \\ -5.1e-13 \\ 6.6e-13}} & 1.0e-14 & 7.35 & 1249 &  & 0.10 & \tiny{\tcell{0.9999999999998730 \\ 2.0000000000000338 \\ 2.9999999999993920 \\ 4.0000000000007159 \\ 4.9999999999995328 \\ 6.0000000000004166 \\ 7.0000000000003917 \\ 7.9999999999994449 \\ 9.0000000000005844 \\ 9.9999999999993232}} & \tiny{\tcell{1.3e-13 \\ -3.4e-14 \\ 6.1e-13 \\ -7.2e-13 \\ 4.7e-13 \\ -4.2e-13 \\ -3.9e-13 \\ 5.6e-13 \\ -5.8e-13 \\ 6.8e-13}} & 9.9e-15 & 8.21 & 1198 \\
\hhline{*{6}{-}~*{6}{-}}
0.20 & \tiny{\tcell{0.9999999999997924 \\ 2.0000000000001532 \\ 2.9999999999992975 \\ 4.0000000000007567 \\ 4.9999999999995675 \\ 6.0000000000003331 \\ 7.0000000000005098 \\ 7.9999999999993401 \\ 9.0000000000006466 \\ 9.9999999999993125}} & \tiny{\tcell{2.1e-13 \\ -1.5e-13 \\ 7.0e-13 \\ -7.6e-13 \\ 4.3e-13 \\ -3.3e-13 \\ -5.1e-13 \\ 6.6e-13 \\ -6.5e-13 \\ 6.9e-13}} & 9.6e-15 & 9.22 & 618 &  & 0.20 & \tiny{\tcell{0.9999999999995949 \\ 2.0000000000004694 \\ 2.9999999999990785 \\ 4.0000000000008127 \\ 4.9999999999997042 \\ 6.0000000000000391 \\ 7.0000000000007976 \\ 7.9999999999991083 \\ 9.0000000000007478 \\ 9.9999999999994067}} & \tiny{\tcell{4.1e-13 \\ -4.7e-13 \\ 9.2e-13 \\ -8.1e-13 \\ 3.0e-13 \\ -3.9e-14 \\ -8.0e-13 \\ 8.9e-13 \\ -7.5e-13 \\ 5.9e-13}} & 9.7e-15 & 10.90 & 569 \\
\hhline{*{6}{-}~*{6}{-}}
0.30 & \tiny{\tcell{0.9999999999996179 \\ 2.0000000000004308 \\ 2.9999999999991189 \\ 4.0000000000007976 \\ 4.9999999999996909 \\ 6.0000000000000782 \\ 7.0000000000007621 \\ 7.9999999999991331 \\ 9.0000000000007425 \\ 9.9999999999993783}} & \tiny{\tcell{3.8e-13 \\ -4.3e-13 \\ 8.8e-13 \\ -8.0e-13 \\ 3.1e-13 \\ -7.8e-14 \\ -7.6e-13 \\ 8.7e-13 \\ -7.4e-13 \\ 6.2e-13}} & 9.7e-15 & 10.63 & 409 &  & 0.30 & \tiny{\tcell{0.9999999999995429 \\ 2.0000000000006515 \\ 2.9999999999993157 \\ 4.0000000000004032 \\ 5.0000000000000773 \\ 5.9999999999996154 \\ 7.0000000000007301 \\ 7.9999999999993188 \\ 9.0000000000004423 \\ 9.9999999999998668}} & \tiny{\tcell{4.6e-13 \\ -6.5e-13 \\ 6.8e-13 \\ -4.0e-13 \\ -7.7e-14 \\ 3.8e-13 \\ -7.3e-13 \\ 6.8e-13 \\ -4.4e-13 \\ 1.3e-13}} & 9.8e-15 & 8.42 & 366 \\
\hhline{*{6}{-}~*{6}{-}}
0.40 & \tiny{\tcell{0.9999999999995401 \\ 2.0000000000006173 \\ 2.9999999999992113 \\ 4.0000000000005675 \\ 4.9999999999999485 \\ 5.9999999999997415 \\ 7.0000000000008020 \\ 7.9999999999991864 \\ 9.0000000000005933 \\ 9.9999999999996767}} & \tiny{\tcell{4.6e-13 \\ -6.2e-13 \\ 7.9e-13 \\ -5.7e-13 \\ 5.2e-14 \\ 2.6e-13 \\ -8.0e-13 \\ 8.1e-13 \\ -5.9e-13 \\ 3.2e-13}} & 9.7e-15 & 9.62 & 308 &  & 0.40 & \tiny{\tcell{0.9999999999996304 \\ 2.0000000000005715 \\ 2.9999999999995834 \\ 4.0000000000001297 \\ 5.0000000000002141 \\ 5.9999999999995408 \\ 7.0000000000005151 \\ 7.9999999999995852 \\ 9.0000000000002007 \\ 10.0000000000000870}} & \tiny{\tcell{3.7e-13 \\ -5.7e-13 \\ 4.2e-13 \\ -1.3e-13 \\ -2.1e-13 \\ 4.6e-13 \\ -5.2e-13 \\ 4.1e-13 \\ -2.0e-13 \\ -8.7e-14}} & 9.7e-15 & 6.20 & 261 \\
\hhline{*{6}{-}~*{6}{-}}
0.50 & \tiny{\tcell{0.9999999999995763 \\ 2.0000000000006102 \\ 2.9999999999993903 \\ 4.0000000000003553 \\ 5.0000000000000906 \\ 5.9999999999996110 \\ 7.0000000000006883 \\ 7.9999999999993481 \\ 9.0000000000004192 \\ 9.9999999999998810}} & \tiny{\tcell{4.2e-13 \\ -6.1e-13 \\ 6.1e-13 \\ -3.6e-13 \\ -9.1e-14 \\ 3.9e-13 \\ -6.9e-13 \\ 6.5e-13 \\ -4.2e-13 \\ 1.2e-13}} & 9.5e-15 & 8.14 & 247 &  & 0.50 & \tiny{\tcell{0.9999999999997562 \\ 2.0000000000004174 \\ 2.9999999999998570 \\ 3.9999999999998987 \\ 5.0000000000002824 \\ 5.9999999999995666 \\ 7.0000000000002665 \\ 7.9999999999998570 \\ 8.9999999999999840 \\ 10.0000000000002292}} & \tiny{\tcell{2.4e-13 \\ -4.2e-13 \\ 1.4e-13 \\ 1.0e-13 \\ -2.8e-13 \\ 4.3e-13 \\ -2.7e-13 \\ 1.4e-13 \\ 1.6e-14 \\ -2.3e-13}} & 9.0e-15 & 4.64 & 196 \\
\hhline{*{6}{-}~*{6}{-}}
0.60 & \tiny{\tcell{0.9999999999996063 \\ 2.0000000000005911 \\ 2.9999999999995000 \\ 4.0000000000002345 \\ 5.0000000000001625 \\ 5.9999999999995524 \\ 7.0000000000006111 \\ 7.9999999999994520 \\ 9.0000000000003180 \\ 9.9999999999999947}} & \tiny{\tcell{3.9e-13 \\ -5.9e-13 \\ 5.0e-13 \\ -2.3e-13 \\ -1.6e-13 \\ 4.5e-13 \\ -6.1e-13 \\ 5.5e-13 \\ -3.2e-13 \\ 5.3e-15}} & 9.5e-15 & 7.21 & 205 &  & 0.60 & \tiny{\tcell{0.9999999999999404 \\ 2.0000000000001767 \\ 3.0000000000002229 \\ 3.9999999999996207 \\ 5.0000000000003348 \\ 5.9999999999996518 \\ 6.9999999999999281 \\ 8.0000000000002061 \\ 8.9999999999997264 \\ 10.0000000000003677}} & \tiny{\tcell{6.0e-14 \\ -1.8e-13 \\ -2.2e-13 \\ 3.8e-13 \\ -3.3e-13 \\ 3.5e-13 \\ 7.2e-14 \\ -2.1e-13 \\ 2.7e-13 \\ -3.7e-13}} & 8.4e-15 & 5.14 & 151 \\
\hhline{*{6}{-}~*{6}{-}}
0.70 & \tiny{\tcell{0.9999999999996703 \\ 2.0000000000005156 \\ 2.9999999999996398 \\ 4.0000000000001084 \\ 5.0000000000002069 \\ 5.9999999999995488 \\ 7.0000000000004858 \\ 7.9999999999995888 \\ 9.0000000000002025 \\ 10.0000000000000853}} & \tiny{\tcell{3.3e-13 \\ -5.2e-13 \\ 3.6e-13 \\ -1.1e-13 \\ -2.1e-13 \\ 4.5e-13 \\ -4.9e-13 \\ 4.1e-13 \\ -2.0e-13 \\ -8.5e-14}} & 8.8e-15 & 6.41 & 175 &  & 0.70 & \tiny{\tcell{1.0000000000004183 \\ 1.9999999999994906 \\ 3.0000000000009850 \\ 3.9999999999991811 \\ 5.0000000000002647 \\ 6.0000000000000453 \\ 6.9999999999991829 \\ 8.0000000000008722 \\ 8.9999999999993072 \\ 10.0000000000004601}} & \tiny{\tcell{-4.2e-13 \\ 5.1e-13 \\ -9.8e-13 \\ 8.2e-13 \\ -2.6e-13 \\ -4.5e-14 \\ 8.2e-13 \\ -8.7e-13 \\ 6.9e-13 \\ -4.6e-13}} & 9.5e-15 & 11.08 & 118 \\
\hhline{*{6}{-}~*{6}{-}}
0.80 & \tiny{\tcell{0.9999999999997253 \\ 2.0000000000004499 \\ 2.9999999999997562 \\ 4.0000000000000115 \\ 5.0000000000002363 \\ 5.9999999999995532 \\ 7.0000000000003819 \\ 7.9999999999997051 \\ 9.0000000000001066 \\ 10.0000000000001581}} & \tiny{\tcell{2.7e-13 \\ -4.5e-13 \\ 2.4e-13 \\ -1.2e-14 \\ -2.4e-13 \\ 4.5e-13 \\ -3.8e-13 \\ 2.9e-13 \\ -1.1e-13 \\ -1.6e-13}} & 8.3e-15 & 5.71 & 152 &  & 0.80 & \tiny{\tcell{1.0000000000001437 \\ 1.9999999999997418 \\ 3.0000000000000266 \\ 4.0000000000001412 \\ 4.9999999999997859 \\ 6.0000000000002691 \\ 6.9999999999999014 \\ 7.9999999999999956 \\ 9.0000000000000817 \\ 9.9999999999998188}} & \tiny{\tcell{-1.4e-13 \\ 2.6e-13 \\ -2.7e-14 \\ -1.4e-13 \\ 2.1e-13 \\ -2.7e-13 \\ 9.9e-14 \\ 4.4e-15 \\ -8.2e-14 \\ 1.8e-13}} & 7.1e-15 & 3.78 & 99 \\
\hhline{*{6}{-}~*{6}{-}}
\cellcolor{orange!30}{0.90} & \tiny{\cellcolor{orange!30}{\tcell{0.9999999999997036 \\ 2.0000000000005018 \\ 2.9999999999997837 \\ 3.9999999999999458 \\ 5.0000000000003126 \\ 5.9999999999994564 \\ 7.0000000000003917 \\ 7.9999999999997238 \\ 9.0000000000000622 \\ 10.0000000000002469}}} & \tiny{\cellcolor{orange!30}{\tcell{3.0e-13 \\ -5.0e-13 \\ 2.2e-13 \\ 5.4e-14 \\ -3.1e-13 \\ 5.4e-13 \\ -3.9e-13 \\ 2.8e-13 \\ -6.2e-14 \\ -2.5e-13}}} & \cellcolor{orange!30}{9.7e-15} & \cellcolor{orange!30}{5.45} & \cellcolor{orange!30}{133} &  & 0.90 & \tiny{\tcell{0.9999999999996279 \\ 2.0000000000005511 \\ 2.9999999999994373 \\ 4.0000000000002709 \\ 5.0000000000000995 \\ 5.9999999999996732 \\ 7.0000000000005222 \\ 7.9999999999995781 \\ 9.0000000000002416 \\ 9.9999999999999964}} & \tiny{\tcell{3.7e-13 \\ -5.5e-13 \\ 5.6e-13 \\ -2.7e-13 \\ -9.9e-14 \\ 3.3e-13 \\ -5.2e-13 \\ 4.2e-13 \\ -2.4e-13 \\ 3.6e-15}} & 9.0e-15 & 6.84 & 80 \\
\hhline{*{6}{-}~*{6}{-}}
\cellcolor{green!30}{0.91} & \tiny{\cellcolor{green!30}{\tcell{0.9999999999997643 \\ 2.0000000000003961 \\ 2.9999999999998503 \\ 3.9999999999999067 \\ 5.0000000000002727 \\ 5.9999999999995275 \\ 7.0000000000002860 \\ 7.9999999999997993 \\ 9.0000000000000071 \\ 10.0000000000002274}}} & \tiny{\cellcolor{green!30}{\tcell{2.4e-13 \\ -4.0e-13 \\ 1.5e-13 \\ 9.3e-14 \\ -2.7e-13 \\ 4.7e-13 \\ -2.9e-13 \\ 2.0e-13 \\ -7.1e-15 \\ -2.3e-13}}} & \cellcolor{green!30}{7.7e-15} & \cellcolor{green!30}{5.62} & \cellcolor{green!30}{132} &  & 0.91 & \tiny{\tcell{0.9999999999996774 \\ 2.0000000000005023 \\ 2.9999999999996070 \\ 4.0000000000001092 \\ 5.0000000000001705 \\ 5.9999999999996536 \\ 7.0000000000003917 \\ 7.9999999999997273 \\ 9.0000000000001137 \\ 10.0000000000000870}} & \tiny{\tcell{3.2e-13 \\ -5.0e-13 \\ 3.9e-13 \\ -1.1e-13 \\ -1.7e-13 \\ 3.5e-13 \\ -3.9e-13 \\ 2.7e-13 \\ -1.1e-13 \\ -8.7e-14}} & 9.5e-15 & 5.17 & 79 \\
\hhline{*{6}{-}~*{6}{-}}
\cellcolor{orange!30}{0.92} & \tiny{\cellcolor{orange!30}{\tcell{0.9999999999999978 \\ 1.9999999999999263 \\ 3.0000000000000084 \\ 3.9999999999998774 \\ 4.9999999999999876 \\ 5.9999999999999289 \\ 6.9999999999998970 \\ 8.0000000000000071 \\ 8.9999999999998863 \\ 10.0000000000000018}}} & \tiny{\cellcolor{orange!30}{\tcell{2.2e-15 \\ 7.4e-14 \\ -8.4e-15 \\ 1.2e-13 \\ 1.2e-14 \\ 7.1e-14 \\ 1.0e-13 \\ -7.1e-15 \\ 1.1e-13 \\ -1.8e-15}}} & \cellcolor{orange!30}{8.9e-15} & \cellcolor{orange!30}{1.27} & \cellcolor{orange!30}{138} &  & 0.92 & \tiny{\tcell{0.9999999999997662 \\ 2.0000000000003832 \\ 2.9999999999997917 \\ 3.9999999999999769 \\ 5.0000000000001945 \\ 5.9999999999996989 \\ 7.0000000000002345 \\ 7.9999999999998801 \\ 9.0000000000000071 \\ 10.0000000000001350}} & \tiny{\tcell{2.3e-13 \\ -3.8e-13 \\ 2.1e-13 \\ 2.3e-14 \\ -1.9e-13 \\ 3.0e-13 \\ -2.3e-13 \\ 1.2e-13 \\ -7.1e-15 \\ -1.4e-13}} & 8.5e-15 & 4.09 & 78 \\
\hhline{*{6}{-}~*{6}{-}}
1.00 & \tiny{\tcell{1.0000000000000275 \\ 2.0000000000000546 \\ 3.0000000000000591 \\ 4.0000000000000622 \\ 5.0000000000000488 \\ 6.0000000000000560 \\ 7.0000000000000542 \\ 8.0000000000000586 \\ 9.0000000000000604 \\ 10.0000000000000586}} & \tiny{\tcell{-2.8e-14 \\ -5.5e-14 \\ -5.9e-14 \\ -6.2e-14 \\ -4.9e-14 \\ -5.6e-14 \\ -5.4e-14 \\ -5.9e-14 \\ -6.0e-14 \\ -5.9e-14}} & 9.7e-15 & 0.91 & 595 &  & 1.00 & \tiny{\tcell{1.0000000000001998 \\ 1.9999999999996729 \\ 3.0000000000001941 \\ 4.0000000000000098 \\ 4.9999999999998446 \\ 6.0000000000002407 \\ 6.9999999999998046 \\ 8.0000000000000924 \\ 8.9999999999999964 \\ 9.9999999999998916}} & \tiny{\tcell{-2.0e-13 \\ 3.3e-13 \\ -1.9e-13 \\ -9.8e-15 \\ 1.6e-13 \\ -2.4e-13 \\ 2.0e-13 \\ -9.2e-14 \\ 3.6e-15 \\ 1.1e-13}} & 7.5e-15 & 3.90 & 68 \\
\hhline{*{6}{-}~*{6}{-}}
& & & & & & & 1.10 & \tiny{\tcell{1.0000000000001097 \\ 1.9999999999997926 \\ 3.0000000000000244 \\ 4.0000000000001226 \\ 4.9999999999998463 \\ 6.0000000000001634 \\ 6.9999999999999671 \\ 7.9999999999999440 \\ 9.0000000000000853 \\ 9.9999999999998810}} & \tiny{\tcell{-1.1e-13 \\ 2.1e-13 \\ -2.4e-14 \\ -1.2e-13 \\ 1.5e-13 \\ -1.6e-13 \\ 3.3e-14 \\ 5.6e-14 \\ -8.5e-14 \\ 1.2e-13}} & 7.1e-15 & 2.76 & 60 \\
\hhline{*{6}{-}~*{6}{-}}
& & & & & & & \cellcolor{orange!30}{1.14} & \tiny{\cellcolor{orange!30}{\tcell{1.0000000000000333 \\ 2.0000000000000209 \\ 3.0000000000002847 \\ 3.9999999999996687 \\ 5.0000000000001688 \\ 5.9999999999999494 \\ 6.9999999999997700 \\ 8.0000000000002878 \\ 8.9999999999997744 \\ 10.0000000000001350}}} & \tiny{\cellcolor{orange!30}{\tcell{-3.3e-14 \\ -2.1e-14 \\ -2.8e-13 \\ 3.3e-13 \\ -1.7e-13 \\ 5.1e-14 \\ 2.3e-13 \\ -2.9e-13 \\ 2.3e-13 \\ -1.4e-13}}} & \cellcolor{orange!30}{8.2e-15} & \cellcolor{orange!30}{4.09} & \cellcolor{orange!30}{58} \\
\hhline{*{6}{-}~*{6}{-}}
& & & & & & & \cellcolor{green!30}{1.15} & \tiny{\cellcolor{green!30}{\tcell{1.0000000000003704 \\ 1.9999999999995046 \\ 3.0000000000007843 \\ 3.9999999999994986 \\ 5.0000000000000195 \\ 6.0000000000002940 \\ 6.9999999999993570 \\ 8.0000000000005471 \\ 8.9999999999996785 \\ 10.0000000000000195}}} & \tiny{\cellcolor{green!30}{\tcell{-3.7e-13 \\ 5.0e-13 \\ -7.8e-13 \\ 5.0e-13 \\ -2.0e-14 \\ -2.9e-13 \\ 6.4e-13 \\ -5.5e-13 \\ 3.2e-13 \\ -2.0e-14}}} & \cellcolor{green!30}{9.9e-15} & \cellcolor{green!30}{7.58} & \cellcolor{green!30}{57} \\
\hhline{*{6}{-}~*{6}{-}}
& & & & & & & \cellcolor{orange!30}{1.16} & \tiny{\cellcolor{orange!30}{\tcell{0.9999999999999343 \\ 2.0000000000000568 \\ 2.9999999999997615 \\ 4.0000000000002212 \\ 4.9999999999999210 \\ 5.9999999999999796 \\ 7.0000000000001972 \\ 7.9999999999997922 \\ 9.0000000000001474 \\ 9.9999999999999361}}} & \tiny{\cellcolor{orange!30}{\tcell{6.6e-14 \\ -5.7e-14 \\ 2.4e-13 \\ -2.2e-13 \\ 7.9e-14 \\ 2.0e-14 \\ -2.0e-13 \\ 2.1e-13 \\ -1.5e-13 \\ 6.4e-14}}} & \cellcolor{orange!30}{4.5e-15} & \cellcolor{orange!30}{5.38} & \cellcolor{orange!30}{59} \\
\hhline{*{6}{-}~*{6}{-}}
& & & & & & & 1.20 & \tiny{\tcell{0.9999999999998656 \\ 2.0000000000001279 \\ 2.9999999999995715 \\ 4.0000000000003775 \\ 4.9999999999998908 \\ 5.9999999999999183 \\ 7.0000000000003659 \\ 7.9999999999996323 \\ 9.0000000000002398 \\ 9.9999999999999325}} & \tiny{\tcell{1.3e-13 \\ -1.3e-13 \\ 4.3e-13 \\ -3.8e-13 \\ 1.1e-13 \\ 8.2e-14 \\ -3.7e-13 \\ 3.7e-13 \\ -2.4e-13 \\ 6.8e-14}} & 8.0e-15 & 5.35 & 60 \\
\hhline{*{6}{-}~*{6}{-}}
& & & & & & & 1.30 & \tiny{\tcell{0.9999999999998268 \\ 2.0000000000001878 \\ 2.9999999999995537 \\ 4.0000000000003508 \\ 4.9999999999999529 \\ 5.9999999999998233 \\ 7.0000000000004077 \\ 7.9999999999996350 \\ 9.0000000000001972 \\ 10.0000000000000284}} & \tiny{\tcell{1.7e-13 \\ -1.9e-13 \\ 4.5e-13 \\ -3.5e-13 \\ 4.7e-14 \\ 1.8e-13 \\ -4.1e-13 \\ 3.7e-13 \\ -2.0e-13 \\ -2.8e-14}} & 8.6e-15 & 5.18 & 79 \\
\hhline{*{6}{-}~*{6}{-}}
& & & & & & & 1.40 & \tiny{\tcell{0.9999999999998201 \\ 2.0000000000002327 \\ 2.9999999999996545 \\ 4.0000000000001883 \\ 5.0000000000000604 \\ 5.9999999999997646 \\ 7.0000000000003144 \\ 7.9999999999997886 \\ 9.0000000000000568 \\ 10.0000000000001386}} & \tiny{\tcell{1.8e-13 \\ -2.3e-13 \\ 3.5e-13 \\ -1.9e-13 \\ -6.0e-14 \\ 2.4e-13 \\ -3.1e-13 \\ 2.1e-13 \\ -5.7e-14 \\ -1.4e-13}} & 6.7e-15 & 5.22 & 147 \\
\hhline{*{6}{-}~*{6}{-}}
& & & & & & & 1.50 & \tiny{\tcell{1.0000000000002334 \\ 1.9999999999996918 \\ 3.0000000000004157 \\ 3.9999999999997984 \\ 4.9999999999998836 \\ 6.0000000000003437 \\ 6.9999999999996145 \\ 8.0000000000002274 \\ 8.9999999999999911 \\ 9.9999999999997407}} & \tiny{\tcell{-2.3e-13 \\ 3.1e-13 \\ -4.2e-13 \\ 2.0e-13 \\ 1.2e-13 \\ -3.4e-13 \\ 3.9e-13 \\ -2.3e-13 \\ 8.9e-15 \\ 2.6e-13}} & 9.6e-15 & 4.62 & 1137 \\
\hhline{*{6}{-}~*{6}{-}}
\end{longtabu}}}

\subsubsection{График зависимости числа итераций от параметра релаксации}

\noindent\begin{tikzpicture}
\begin{semilogyaxis}[xlabel=w,ylabel=Итераций,width=\textwidth, height=6cm]
\addplot[red, no markers] table [y=it1, x=w1]{B.dat};
\addplot[blue, no markers] table [y=it2, x=w2]{B.dat};
\legend{Метод Якоби,Метод Зейделя}
\end{semilogyaxis}
\end{tikzpicture}

\subsubsection{Расчет числа обусловленности через MathCad}
$\displaystyle
	\begin{aligned}
		&\mathop{conde}(B) = 44.931 \\
		&\mathop{condi}(B) = 34.897 \\
		&\mathop{cond1}(B) = 53.051  \\
		&\mathop{cond2}(B) = 20.536  \\
		&\frac{\lambda^B_{max}}{\lambda^B_{min}} = \frac{14.284}{0.93} = 15.359  \\
	\end{aligned}
$

\section{Выводы}

Исследования показали, что для различных матриц необходим различный параметр релаксации, и что иногда он может лежать за допустимыми пределами (как это было для метода Якоби, где $w = 1.08$). График зависимости числа итераций от параметра релаксации имеет один ярко выраженный минимум, что позволяет подобрать его благодаря методам поиска минимума, либо различным эвристикам.

Так же было оценено число обусловленности по относительной невязке и погрешности: $ \mathop{cond}(A) > 8.54 $, $ \mathop{cond}(B) > 10.90 $. Смотря на расчет числа обусловленности через специальные программы, можем заметить, что в реальности оно в несколько раз больше, чем было оценено.