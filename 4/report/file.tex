\mytitlepage{прикладной математики}{4}{Численные методы}{Решение систем нелинейных уравнений методов Ньютона}{ПМ-63}{Шепрут И.И.}{Все}{Задорожный А.Г.}{2018}

\section{Цель работы}

Разработать программу решения системы нелинейных уравнений (СНУ) методом Ньютона. Провести исследования метода для нескольких систем размерности от 2 до 10.

\section{Исследования}

\textit{Краткое описание визуализации:} невязка в каждой точке рисуется после приведения СНУ к квадратному виду, сделано это для большей наглядности, потому что норма невязки от всех $m>n$ уравнений не настолько точно показывает куда будет двигаться метод. Так же, из-за того, что изображение получалось слишком светлым возле точки решения, невязка нормируется и возводится в квадрат, поэтому изображения стали более темными.

\subsection{Одна окружность}
\subsubsection{Аналитическое вычисление матрицы Якоби}
\subsubsection{Численное вычисление матрицы Якоби}
\subsection{Две окружности}
\subsubsection{Не пересекаются}
\subsubsubsection{Аналитическое вычисление матрицы Якоби}
\subsubsubsection{Численное вычисление матрицы Якоби}
\subsubsection{Пересекаются в одной точке}
\subsubsubsection{Аналитическое вычисление матрицы Якоби}
\subsubsubsection{Численное вычисление матрицы Якоби}
\subsubsection{Начальное приближение лежит на оси симметрии}
\subsubsubsection{Аналитическое вычисление матрицы Якоби}
\subsubsubsection{Численное вычисление матрицы Якоби}
\subsubsection{Начальное приближение лежит на оси, соединяющей центры окружностей}
\textit{Комментарий:} добавлено немного смещения, потому что на на этой оси метод не сходится.
\subsubsubsection{Аналитическое вычисление матрицы Якоби}
\subsubsubsection{Численное вычисление матрицы Якоби}
\subsubsection{Начальное приближение лежит в центре одной из окружностей}
\textit{Комментарий:} добавлено немного смещения, потому что на на этой оси метод не сходится.
\subsubsubsection{Аналитическое вычисление матрицы Якоби}
\subsubsubsection{Численное вычисление матрицы Якоби}
\subsubsection{Добавлена ещё прямая}
\subsubsubsection{Аналитическое вычисление матрицы Якоби}
\subsubsubsection{Численное вычисление матрицы Якоби}
\subsubsection{Пересекаются в двух точках}
\subsubsubsection{Аналитическое вычисление матрицы Якоби}
\subsubsubsection{Численное вычисление матрицы Якоби}
\subsection{Три попарно пересекающиеся прямые}
\subsubsection{Невзвешенный вариант}
\subsubsubsection{Аналитическое вычисление матрицы Якоби}
\subsubsubsection{Численное вычисление матрицы Якоби}
\subsubsection{Взвешенный вариант}
\subsubsubsection{Аналитическое вычисление матрицы Якоби}
\subsubsubsection{Численное вычисление матрицы Якоби}
\subsection{Прямая и синусоида}
\subsubsection{Аналитическое вычисление матрицы Якоби}
\subsubsection{Численное вычисление матрицы Якоби}

\subsection{Влияние размера шага при численном вычислении производной на сходимость метода}

$$ \varepsilon = 10^{-14}, \quad iterations_{max} = 10000, \quad start = \begin{pmatrix} 0 & 0 & 0 & 0 & 0 & 0 & 0 & 0 & 0 & 0 \end{pmatrix}^T $$

\section{Выводы}

\noindent\begin{easylist}
\ListProperties(Hang1=true, Margin2=12pt, Style1**=$\bullet$ , Hide2=1, Hide1=1)
& По таблицам видно, что использование предобусловливания, даже диагонального, положительно влияет на скорость сходимости. 
& В среднем самый быстрый метод - ЛОС LU(sq). 
& Диагональное предобуславливание увеличивает скорость сходимости для МСГ больше, чем для ЛОС.
\end{easylist}

\section{Код программы}

\begin{multicols*}{2}
\mycodeinput{c++}{../logic.h}{logic.h}
\mycodeinput{c++}{../logic.cpp}{logic.cpp}
\end{multicols*}